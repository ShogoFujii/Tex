%%%
%% dvips -Pprinter -t a4 -O 0mm,0mm chklist.dvi
%% [2003/12/05]
\documentclass{jarticle}
\usepackage{amssymb}

\headsep0mm
\headheight0mm
\footskip0mm
\textheight=260mm % 25.4-(297-260)/2=6.9
\textwidth=180mm  % 25.4-(210-180)/2=10.4
\voffset -6.9mm
\hoffset -10.4mm
\oddsidemargin=0mm
\topmargin=0mm
\baselineskip=5mm

\makeatletter
\@ifundefined{inhibitglue}{\def\inhibitglue{\kern-.5zw}}{}
\renewenvironment{enumerate}
  {\ifnum \@enumdepth >\thr@@\@toodeep\else
   \advance\@enumdepth\@ne
   \edef\@enumctr{enum\romannumeral\the\@enumdepth}%
   \list{\csname label\@enumctr\endcsname}{%
         \ifnum\@enumdepth=\@ne
          \topsep\baselineskip\parsep\z@\partopsep\z@\itemsep.5\baselineskip
          \leftmargin1.5zw
          \labelwidth1zw
          \labelsep.5zw
         \else
          \topsep\z@\parsep\z@\partopsep\z@\itemsep\z@
          \leftmargin2zw
          \labelwidth1.25zw
          \labelsep.75zw
         \fi
         \usecounter{\@enumctr}%
         \def\makelabel##1{\hss\llap{##1}}}%
   \fi}{\endlist}
\renewcommand{\theenumii}{\@arabic\c@enumii}
\renewcommand{\labelitemi}{\leavevmode
 \raise.1zh\hbox to 1zw{\hss \textbullet\hss}}
\renewenvironment{itemize}
  {\ifnum \@itemdepth >\thr@@\@toodeep\else
   \advance\@itemdepth\@ne
   \edef\@itemitem{labelitem\romannumeral\the\@itemdepth}%
   \expandafter
   \list{\csname \@itemitem\endcsname}{%
         \topsep\z@\parsep\z@\partopsep\z@\itemsep\z@
         \labelwidth1zw
         \labelsep\z@
         \leftmargin1zw
         \listparindent1zw
         \def\makelabel##1{\hss\llap{##1}}}%
   \fi}{\endlist}
\makeatother

\begin{document}
\pagestyle{empty}

\begin{center}
{\LARGE 投稿者チェックリスト}
\end{center}

{\bfseries 本表で論文の投稿状況がわかるように,
下記のすべての項目に御回答下さい.
御回答なき場合は,受付原稿の事務処理が遅れることもありますので
御留意下さい.}

\begin{enumerate}
\item
他誌との投稿について:
下記の内容について,著者全員で確認されていますか?(□はい□いいえ)\par

\inhibitglue
【本論文は,それと同一内容または極めて類似した内容のものが
同一著者もしくはその中の少なくとも1名を含む著者によって
本会の他の論文誌,他の学術論文誌あるいは商業雑誌に掲載済み,
または投稿中ではない.】

\hfill
(投稿のしおり1.2項「執筆上の注意事項」より抜粋)

\item
著者全員の氏名と会員種別
(名誉員,正員:フェロー,正員,学生員,准員,非会員),
会員番号を御記入下さい.

\inhibitglue
(例)大阪 花子(正員,0000236)\par
\hfill (\hskip5zw,\hskip5zw)\hfill (\hskip5zw,\hskip5zw)\hfill 
(\hskip5zw,\hskip5zw)\par
\hfill (\hskip5zw,\hskip5zw)\hfill (\hskip5zw,\hskip5zw)\hfill 
(\hskip5zw,\hskip5zw)

\item
論文題名:

\mbox{}

\item
連絡先\par
氏名(ふりがな):\par
機関・部署(学生の方は研究室名も御記入下さい):\par
住所:〒\par
電話:\hskip13zw FAX:\hskip13zw E-mail: 

\item
本論文に対応する掲載希望分冊と論文種別について,
□に $\checkmark$ を御記入下さい.\par

\vskip.5\baselineskip

\tabcolsep.5zw
\begin{tabular}{@{}p{7zw}p{8zw}p{7zw}p{8zw}p{7zw}p{9zw}@{}}
□和文誌A
 & & & & & \\
\hskip3zw □論文
 & & & & \\
\hskip3zw □レター
 & (□研究速報
  & □紙上討論
   & □問題提起
    & □訂正)
     & \\
□和文誌B
 & & & & & \\
\hskip3zw □論文
 & (□理論\hskip1zw □実験
  & □理論と実験
   & \multicolumn{2}{l}{□システム開発・ソフトウェア開発}
     & □サーベイ論文)\\
\hskip3zw □レター
 & (□研究速報
  & □紙上討論
   & □問題提起
    & □訂正)
     & \\
□和文誌C
 & & & & & \\ 
\hskip3zw □論文
 & & & & & \\
\hskip3zw □レター
 & □ショートノート
  & □紙上討論
   & □問題提起
    & □訂正
     & \\
\multicolumn{6}{l}{□和文誌DI\hskip1zw □和文誌DII} \\
\hskip3zw □論文
 & (□一般論文
  & □システム開発
   & □サーベイ論文)
    &
     & \\
\hskip3zw □レター
 & (□研究速報
  & □紙上討論
   & □問題提起
    & □訂正)
     & \\
\end{tabular}

\item
査読を受ける希望の分野を御記入下さい.
(付録H「専門分野分類表」から優先順に記入.
項目5で選んだ分冊と対応させること)

\inhibitglue
(例)□和文論文誌C\hskip1zw □論文\hskip1zw (1) 分類番号[C010100]\hskip2zw
分類名[電磁界理論]\par
\begin{enumerate}
\item
分類番号[\hskip3zw ]分類名[\hskip30zw ]
\item
分類番号[\hskip3zw ]分類名[\hskip30zw ]
\item
分類番号[\hskip3zw ]分類名[\hskip30zw ]
\end{enumerate}

*「専門分野分類表」にない場合には,分類番号の欄に[F999999]と記入し,
分類名を御記入下さい.

\item
本論文が過去に不採録となった論文,あるいは,条件付き採録の
判定を受けながら自ら取り下げた論文を修正したものであれば,
その論文の受付番号を記して下さい.

\hfill
\underline{受付番号:\hspace*{10zw}}

\item
著者想定ページ数
\begin{itemize}
\item
刷上りページ数を何ページと想定していますか?\hskip3zw 
  刷上りページ数(\hskip3zw )ページ
\item
基準のページ数を超えた場合,掲載別刷代を支払い可能ですか.\hskip2zw 
  □はい\hskip2zw □いいえ
\end{itemize}
\end{enumerate}

\end{document}
