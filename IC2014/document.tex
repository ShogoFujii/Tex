\documentclass[11pt, a4paper, twocolumn]{jsarticle}
%
\usepackage{amsmath,amssymb}
\usepackage{bm}
\usepackage{graphicx}
\usepackage{ascmac}
\usepackage{subfigure}
\usepackage{multicol}
\usepackage{setspace}
\usepackage{mediabb}
\usepackage{float}
\usepackage{latexsym}
\usepackage{url}
\usepackage{cite}
%
\setlength{\textwidth}{\fullwidth}
\setlength{\textheight}{35\baselineskip}
\addtolength{\textheight}{\topskip}
\setlength{\voffset}{-0.5in}
\setlength{\headsep}{0.3in}
%
\renewcommand{\baselinestretch}{0.9}
\renewcommand{\figurename}{図}
\renewcommand{\tablename}{表}
\renewcommand{\headfont}{\bfseries}

\pagestyle{empty}

\begin{document}
% \twocolumn[
% \begin{screen}
% \begin{flushleft}
% aa\hspace{\fill}yyyy/mm/dd
% \end{flushleft}
% \begin{center}
% {\Large  和文タイトル  \\
% 英文タイトル}
% \end{center}
% \begin{flushleft}
% 工学系研究科 電気系工学専攻 関谷研究室\hspace{\fill}修士課程1年 37-136482 藤居 翔吾
% \end{flushleft}
% \end{screen}
% ]

\twocolumn[
\vspace{1.5cm}
\begin{center}
{\Large データセンター環境におけるショートフロー通信改善手法の一提案}\\
\vspace{1em}
{\large $藤居\;翔吾^{\dagger}$ \hspace{1.0cm}$田崎\;創^{\ddagger}$ \hspace{1.0cm}
$関谷\;勇司^{\ddagger}$}\\
${}^{\dagger}$東京大学大学院工学系研究科\\
${}^{\ddagger}$東京大学情報基盤センター \\
\end{center}
\begin{quotation}
\begin{spacing}{0.6}
{\footnotesize クラウド型のサービスの性質により, 今日のデータセンターではデータセンター内のトラフィックが増大しており,
既存のTCPを拡張したMultipath TCP(MPTCP)により, 通信性能向上を目指す取り組みがされている.
しかし, 様々なニーズを抱えるトラフィックが混在している中で, レイテンシ志向なサイズの小さいフローに対し,
既存のMPTCP実装では性能を劣化させる問題が報告されている.
そこで本論文では, この問題に対し, 並列分散処理アプリケーションが稼働しているクラスターPCのトラフィックを観測する事で, 単一NIC(Network
Interface Card)への通信集中による遅延の影響がある事を示し,
MPTCPによるデータセンターモデルのような複数のNICを持ち, それぞれのノードが複数のキューを持つ環境で,
通信経路を切り替えることによりキューの負荷を分散させる手法を提案した.
この手法による効果について, 中継スイッチとエンドノードに対してスループット, フロー完結時間の二つのメトリックを用いた予備検証を行い,
改善手法が与える影響を考察する.
}
\end{spacing}
\end{quotation}

\begin{center}
{\Large A proposal method for shortflow traffic in datacenter network }\\
\vspace{1em}
{\large ${\rm Shogo\;Fujii^{\dagger}}$
\hspace{1.0cm}${\rm Hajime\;Tazaki^{\ddagger}}$ \hspace{1.0cm}
${\rm Yuji\;Sekiya^{\ddagger}}$}\\
${}^{\dagger}$The University of Tokyo, Graduate School of Engineering\\
${}^{\ddagger}$The University of Tokyo, Information Technology
Center \\
\end{center}
\begin{quotation}
\begin{spacing}{0.6}
{\footnotesize
As increasing the amount of traffic in a datacenter by cloud service, the
effective network for utilization of massive computer clusters has been studied.
Recently, Multipath TCP(MPTCP) has been tackled this problem.
MPTCP can achieve the effective consumptions of the resources with multipath,
but a researcher reported MPTCP causes the delay of flow completion for short
flows.
In this paper, I presented measurements of the distribute processing cluster PCs
and reveal impairment mechanisms that lead to that latencies, rooted in single
NIC(Network Interface Card) with intensive load traffic, and proposed the method balancing
the load of queue for multiple queue, in such a MPTCP datacenter model.
I verified the effect of the method for switch and end-node with two metrics,
FCT(Flow completion time) of short flows and throughput of backgroung flows,
and considered the effects of the load balancing method as preliminary
experiment. }
\end{spacing}
\end{quotation}

\vspace{1.5cm}
]


\section{まえがき}


\section{関連研究}
\label{sec:related}


\section{データセンターネットワーク}
\label{sec:datacenter}

\subsection{マルチパス環境を実現するネットワークトポロジー}

\subsection{トラフィックシナリオ}
\label{sec:fattree}
この節では, データセンターを構成する要素について述べる.
\subsubsection{実トラフィック解析}
\label{sec:traffic_character}

\section{提案手法}
\label{sec:proposed_method}

\subsection{キュー負荷分散}
\label{sec:load_balancing_mechanism}

\subsection{期待される効果}
\label{sec:expected_effect}


\section{検証実験}
\label{sec:verification}

\subsection{実験環境}


\subsection{実験結果}

\subsection{考察}
\label{sec:analysis}

\section{あとがき}
\label{sec:conclude}


\begin{spacing}{0.7}
\footnotesize{
\begin{thebibliography}{99}% 文献数が10未満の時 {9}
\bibitem{IBM_rep}{日本アイ・ビー・エム株式会社. IBM 第1章
大容量データのバックアップ,
\url{http://www-06.ibm.com/systems/jp/storage/column/backup/01.html}}
\bibitem{amazon}{Jim Liddle. Amazon found every 100ms of latency cost them 1\%
in sales, August 2008.
\url{http://blog.gigaspaces.com/amazon-found-every-100ms-of-latency-cost}

\url{-them-1-in-sales/}}

\bibitem{customer_impact}{R. Kohavi et al. Practical Guide to Controlled
Experiments on theWeb: Listen to Your Customers not to the HiPPO. KDD, 2007.}
\bibitem{requirement}{J. Hamilton. On designing and deploying Internet-scale
services. In USENIX LISA, 2007.}
\bibitem{presto}{Facebook. Presto: Interacting with petabytes
of data at Facebook,
\url{https://www.facebook.com/notes/facebook-engineering/presto-interacting-with-petabytes-of-data}

\url{-at-facebook/10151786197628920}}
\bibitem{mapreduce}{Dean, Jeffrey, and Sanjay
Ghemawat. "MapReduce: simplified data processing on large clusters." Communications of the ACM 51.1 (2008): 107-113.} \bibitem{dryad}{Isard, Michael, et al. "Dryad: distributed data-parallel
programs from sequential building blocks." ACM SIGOPS Operating Systems Review 41.3 (2007): 59-72.}
\bibitem{fattree}{Al-Fares, Mohammad, Alexander Loukissas, and Amin Vahdat. "A
scalable, commodity data center network architecture." ACM SIGCOMM Computer Communication Review. Vol. 38. No. 4. ACM, 2008.}
\bibitem{bcube}{Guo, Chuanxiong, et al. "BCube: a high performance,
server-centric network architecture for modular data centers." ACM SIGCOMM Computer Communication Review 39.4 (2009): 63-74.}
\bibitem{vl2}{Greenberg, Albert, et al. "VL2: a scalable and flexible data
center network." ACM SIGCOMM Computer Communication Review. Vol. 39. No. 4. ACM, 2009.}
\bibitem{dctcp}{Alizadeh, Mohammad, et al. "Data center tcp (dctcp)." ACM SIGCOMM Computer Communication Review 40.4 (2010): 63-74.}
\bibitem{improving}{Raiciu, Costin, et al. "Improving datacenter performance and
robustness with multipath TCP." ACM SIGCOMM Computer Communication Review. Vol. 41. No. 4. ACM, 2011.}
\bibitem{detail}{Zats, David, et al. "DeTail: Reducing the flow completion time
tail in datacenter networks." ACM SIGCOMM Computer Communication Review 42.4 (2012): 139-150.}
\bibitem{p_fab}{Alizadeh, Mohammad, et al. "pfabric: Minimal near-optimal datacenter transport." Proceedings of the ACM SIGCOMM 2013 conference on SIGCOMM. ACM, 2013.}
\bibitem{click}{Kohler, Eddie, et al. "The Click modular router." ACM
Transactions on Computer Systems (TOCS) 18.3 (2000): 263-297.}
\bibitem{mptcp}{Ford, Alan, et al. TCP Extensions for Multipath Operation with
Multiple Addresses: draft-ietf-mptcp-multiaddressed-03. No. Internet draft (draft-ietf-mptcp-multiaddressed-07). Roke Manor, 2011.}
\bibitem{cong}{Raiciu, C., M. Handley, and D. Wischik. "Coupled congestion
control for multipath transport protocols." draft-ietf-mptcp-congestion-01 (work in progress) (2011).}
\bibitem{ns3}{Inria「DCE - GETTING STARTED Direct Code Execution」
\url{http://www.nsnam.org/~thehajime/ns-3-dce-doc/getting-started.html}}
\bibitem{traffic}{Benson, Theophilus, Aditya Akella, and David A. Maltz.
"Network traffic characteristics of data centers in the wild." Proceedings of the 10th ACM SIGCOMM conference on Internet measurement. ACM, 2010.}
\bibitem{NAPI}{J. Salim, When NAPI Comes to Town, Proceedings of Linux 2005
Conference, UK, August 2005.}
\bibitem{RSS}{Microsoft corporation. scalable networking with rss, 2005.}
\bibitem{RFS}{Herbert, T. rfs: receive flow steering, september 2010.
http://lwn.net/Articles/381955/.}
\bibitem{RPS}{Herbert, T. rps: receive packet steering, september 2010.
http://lwn.net/Articles/361440/.}
\bibitem{mptcp_linux}{ip networking lab「MultiPath
TCP - Linux Kernel implementation」\url{http://mptcp.info.ucl.ac.be/}}
\bibitem{rtt}{Vasudevan, Vijay, et al. "Safe and effective fine-grained TCP
retransmissions for datacenter communication." ACM SIGCOMM Computer Communication Review. Vol. 39. No. 4. ACM, 2009.}
\bibitem{mptcp_ana}{藤居 翔吾, 田崎 創, 関谷 勇司, "MultiPath TCP
適用時のデータセンターネットワークでのフローサイズが与える影響に関する一考察", 電子情報通信学会, 信学技法, vol. 113, no. 364,
IA2013-65, pp. 47-52, 2013.}
\bibitem{flexible}{P. Agarwal, B. Kwan, and L. Ashvin. Flexible buffer allocation entities for
traffic aggregate containment. US Patent 20090207848, August 2009.}
\bibitem{synchro}{S. Floyd and V. Jacobson. The synchronization of periodic routing messages.
IEEE/ACM ToN, 1994.}
\bibitem{balia}{A. Walid, et al. Balanced Linked Adaptation Congestion Control
Algorithm for MPTCP draft-walid-mptcp-congestion-control-00, 2014.}

\end{thebibliography}
}
\end{spacing}


\end{document}